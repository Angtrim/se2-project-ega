%!TEX root = ../main.tex
\section{Tasks} % (fold)
\label{sec:tasks}

The fulfillment of the project requires a list of tasks to be completed in the order they are presented here, it is important to note, however that this list of tasks is intended to be used as a general guideline and all the tasks may receive updates along the way.
Small changes can be made to the order, especially the last two and the first two steps that will be re-iterated over and over as new requirements emerge.
So, the first version of each task will have strict deadlines, so that the following task can start, while the re-iterations will have specific deadlines that will be specified along the process.

Here is the a table that summirises the main tasks to be completed:
\begin{center}
\begin{tabular}{ |c|l| } \hline
	\textbf{Order} & \textbf{Task} \\ \hline
	1 & RASD \\ \hline
	2 & DD \\ \hline
	3 & ITPD \\ \hline
	4 & PPD \\ \hline
	5 & Implementation \& Unit Testing \\ \hline
	6 & Integration Testing \\ \hline
\end{tabular}
\end{center}

Here we explain what each task is for:

\begin{enumerate}
	\item Prepare a document that specifies the goals of the system, the assumptions that had been made, and the requirements of the \emph{system-to-be}, namely the RASD.
	\item Prepare a document that specifies the design and architecture of the \emph{system-to-be}, namely the DD.
	\item Prepare a document that specifies the plan to follow in order to perform the integration testing, namely the ITPD.
	\item Prepare a document that specifies the plan of the project, namely the PPD.
	\item Write the software, according to what has been specified in the documents produces during the previous tasks, along with the unit testing.
	\item Perform the integration testing, following the plan described in the ITPD.
\end{enumerate}

\subsection{Schedule} % (fold)
\label{sub:schedule}

In this subsection we present a schedule that must followed, however, as always, small changes can always be made.

We explain the schedule through the Gantt chart that follows:

\begin{figure}[!h]
    \hspace*{-2.0cm}
    \begin{ganttchart}[
    		vgrid={*{6}{draw=none}, dotted},
    		hgrid={dotted},
    		time slot format=isodate,
			x unit=0.5mm,
			progress=today,
			progress label text={#1}\%,
	        today=2016-02-02,
	        today rule/.style= {gray!80, thick},
	        today label=Today,
	        today label/.style = {black},
	        link bulge=1, link tolerance=0,
    	]{2015-10-10}{2016-06-20}
        \gantttitlecalendar{year, month} \\
        \ganttbar[name=rasd, progress=100]{RASD}{2015-10-15}{2015-11-06} \\
        \ganttbar[name=dd, progress=100]{DD}{2015-11-12}{2015-12-04} \\
        \ganttbar[name=itpd, progress=100]{ITPD}{2016-01-07}{2016-01-21} \\
        \ganttbar[name=pp, progress=100]{PPD}{2016-01-22}{2016-02-02} \\
        \ganttbar[name=impl, progress=0]{Implementation}{2016-02-03}{2016-06-01} \\
        \ganttbar[name=int-test, progress=0]{Integration Test}{2016-06-02}{2016-06-15}
        \ganttlink[link/.append style=gray!90]{rasd}{dd}
        \ganttlink[link/.append style=gray!90]{dd}{itpd}
        \ganttlink[link/.append style=gray!90]{impl}{int-test}
    \end{ganttchart}
    \label{tab:gantt}
\end{figure}
% subsection schedule (end)

Since the precise deadlines are not legible from the Gantt chart, we give also a table summarising the main deadlines.

\begin{center}
\begin{tabular}{ |c|c|l|c| } \hline
	\textbf{Order} & \textbf{Begin date} & \textbf{Task} & \textbf{Deadline} \\ \hline
	1 & 15-10-2015 & RASD & 06-11-2015 \\ \hline
	2 & 12-11-2015 & DD & 04-12-2015 \\ \hline
	3 & 07-01-2016 & ITPD & 21-01-2016 \\ \hline
	4 & 22-01-2016 & PPD & 02-02-2016 \\ \hline
	5 & 03-02-2016 & Implementation \& Unit Testing  & 01-06-2016 \\ \hline
	6 & 02-06-2016 & Integration Testing  & 15-06-2016 \\ \hline
\end{tabular}
\end{center}


% section tasks (end)