\documentclass[12pt, a4paper]{article}
\begin{document}

\section{Overall Description}
\subsection{Product Perspective}

The product we will release is composed by four main software applications.
\begin{itemize}
  \item A \textbf{Web Application (User)} addressed to the \emph{users}  to use our service. This application has to interface mainly with the \textbf{Back-end Application} and with Google's Maps API.
  \item A \textbf{Mobile Application (User)} addressed to the \emph{users} to use our service and available for Android, iOS, Windows Mobile and Blackberry. This application has to interface mainly with our \textbf{Back-end Application} and with Google's Maps API
  \item A \textbf{Mobile Application (Taxi Driver)} addressed to the \emph{taxi drivers} to use our service and available for Android, iOS. This application has to interface mainly with our \textbf{Back-end Application} and with Google's Maps API

  \item A \textbf{Back-end Application} that will handle all the business logic and that has to interface mainly with Google's Maps API and with a MongoDB database .
\end{itemize}

\subsubsection{User Interfaces}
%to complete
\subsubsection{Hardware Interfaces}
Both the \emph{Mobile Application (User)} and the \emph{Mobile Application (Taxi Driver)} has to interface with the GPS module and with the Network module.\\

\subsubsection{Software Interfaces}


\paragraph{Software interfaces for the Web Application (User)} 

\begin{itemize}
	\item MyTaxyService API
	\begin{itemize}
		\item Mnemonic : Back-end
	\end{itemize}
	\item Google Maps API
	\begin{itemize}
		\item Mnemonic : Google Maps API
		\item Version Number : V3
		\item Source : \emph{https://developers.google.com/maps/documentation/javascript/} 
	\end{itemize}

\end{itemize}

\paragraph{Software interfaces for the Mobile Application (User and Taxi Driver)} 
\begin{itemize}
	\item MyTaxyService API
	\begin{itemize}
		\item Mnemonic : Back-end
	\end{itemize}
	\item Google Maps API
	\begin{itemize}
		\item Mnemonic : Google Maps API
		\item Version Number : V3
		\item Source : \emph{https://developers.google.com/maps/} 
	\end{itemize}

	\item Android SDK
	\begin{itemize}
		\item Mnemonic : Android
		\item Version Number : 6.0
		\item Source : \emph{http://developer.android.com/sdk/index.html} 
	\end{itemize}

	\item iOS SDK
	\begin{itemize}
		\item Mnemonic : iOS
		\item Version Number : 9.2
		\item Source : \emph{https://developer.apple.com/ios/download/} 
	\end{itemize}

	\item Windows Mobile SDK
	\begin{itemize}
		\item Mnemonic : Windows Mobile
		\item Version Number : 6.5
		\item Source : \emph{http://www.microsoft.com/en-us/download/details.aspx?id=17284} 
	\end{itemize}

	\item BlackBerry SDK
	\begin{itemize}
		\item Mnemonic : BlackBerry
		\item Version Number : 10
		\item Source : \emph{https://developer.blackberry.com/} 
	\end{itemize}		
\end{itemize}

\paragraph{Software interfaces for the Back-end Application} 

\begin{itemize}
	\item Node.js API
	\begin{itemize}
		\item Mnemonic : Node.js
		\item Version : 4.2.1
		\item Source : \emph{https://nodejs.org/api/}
	\end{itemize}

	\item MongoDB API
	\begin{itemize}
		\item Mnemonic : MongoDB
		\item Version : 3.0
		\item Source : \emph{https://docs.mongodb.org/manual/}
	\end{itemize}

	\item Google Maps API
	\begin{itemize}
		\item Mnemonic : Google Maps API
		\item Version Number : V3
		\item Source : \emph{https://developers.google.com/maps/documentation/javascript/}
	\end{itemize}
	\item Javascript API
	\begin{itemize}
		\item Mnemonic : Javascript
		\item Source : \emph{https://developer.mozilla.org/en/docs/Web/API} 
	\end{itemize}
\end{itemize}

\subsubsection{Communication Interfaces} 
\label{ssub:communication_interfaces}
\paragraph{Web Application (User)}\mbox{} \\
Every application has to interface with the Internet network. This interface is handled by the operative systems and not by the applications themselves.

\subsubsection{Memory Constrains} % (fold)

\label{ssub:memory_constrains}
\paragraph{Mobile Applications (User and Taxi Driver) and Web Application}
The Mobile and Web Applications can not exceed 75MB of RAM usage.
\paragraph{Back-End Application}
The Back-End application can not exceed 15GB of total RAM usage.

\subsubsection{Site Adaptation Requirements} % (fold)
\label{ssub:site_adaptation_requirements}
\paragraph{Software Adaptation} \mbox{} \\
Every Mobile Application has to be developed according to the platforms design guidelines.










\end{document}