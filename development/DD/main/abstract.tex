%!TEX root = ../main/main.tex
\begin{abstract}

The current document represents the \emph{Software Design Document} (SDD) of \emph{myTaxiService} system.
It provides a representation of the system's framework, based upon the description of the system illustrated in the \emph{Requirement and Analysis Specification Document} (RASD).
Throughout the document a comprehensive study is conducted, providing many different views of the architecture, in order to offer as many perspectives as possible. This is to ensure that any detail of the architecture is thoroughly analyzed.

The main stakeholders of this document are the teams in charge of the implementation and the software maintenance, for this reason a wide gamut of diagrams is offered, ranging from \emph{class diagrams} to \emph{ER diagrams} and \emph{architecture diagrams}.




% This document is intended to be read and used by everyone involved with the development of the software (i.e. developers, testers, project managers) and the delegating institution (i.e. the government of the large city). Therefore its purpose is to provide the intended reader with a fulfilling guideline, that can be used as the main resource for the development of the system, and with a presentation of the challenges that will have to be addressed throughout the whole process.

% An incredible amount of effort has been put into every section of this document; every aspect of the \emph{system-to-be} has been thoroughly explored, meticulously analyzed and accurately studied so that every situation can be addressed in a unique way and can be interpreted without ambiguity.

% In the following sections the system is broken down into its components, examining its functional and non-functional requirements. An in depth analysis of the relationship between the system and the external world is conducted throughout the document, stressing the assumptions that have been taken and highlighting the constraints that every requirement has to satisfy in order to have a functional and useful system.


% The reader of this document will find UML class diagrams, sequence diagrams and use cases that provide all the needed information required to build the system.
% Furthermore, every use case is justified by the goals and assumptions that lead to the need of a specific use case.

% Moreover, the delegating institution will find a reason for every design choice taken and therefore is able understand the necessity of a specific feature.

% This way, each and every reader will have a clear understanding of the \emph{system-to-be}, by finding a justification for every decision taken concerning any aspect that is worth of considering for a successful development of the investigated system.
\end{abstract}
\newpage

