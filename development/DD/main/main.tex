\documentclass[12pt, a4paper]{article}
\usepackage[dvipsnames]{xcolor}
\usepackage{listings}
\usepackage{enumitem, lmodern}
\usepackage[bottom]{footmisc}
\usepackage{import}
\usepackage{caption}
\usepackage{subcaption}
\usepackage{tocbibind}
\usepackage[bookmarks, breaklinks]{hyperref} 
\usepackage{graphicx}
\usepackage{mwe}
\usepackage[toc,page]{appendix}
\usepackage{pdflscape}

\setcounter{tocdepth}{5}
\makeatletter
\renewcommand\paragraph{\@startsection{paragraph}{4}{\z@}%
                                     {-3.25ex\@plus -1ex \@minus -.2ex}%
                                     {0.0001pt \@plus .2ex}%
                                     {\normalfont\normalsize\bfseries}}
\renewcommand\subparagraph{\@startsection{subparagraph}{5}{\z@}%
                                     {-3.25ex\@plus -1ex \@minus -.2ex}%
                                     {0.0001pt \@plus .2ex}%
                                     {\normalfont\normalsize\bfseries}}
\makeatother

\graphicspath{ {../images/} }
\linespread{1}
\begin{document}

\begin{titlepage}

\newcommand{\HRule}{\rule{\linewidth}{0.5mm}} % Defines a new command for the horizontal lines, change thickness here

\center % Center everything on the page
%----------------------------------------------------------------------------------------
%	LOGO SECTION
%----------------------------------------------------------------------------------------

\includegraphics[width=0.25\paperwidth]{logo}\\[0.5cm] % Include a department/university logo - this will require the graphicx package
 
%----------------------------------------------------------------------------------------
%	HEADING SECTIONS
%----------------------------------------------------------------------------------------

\textsc{\LARGE Politecnico di Milano}\\[0.4cm] % Name of your university/college
\textsc{\Large Scuola di Ingegneria Industriale e dell'Informazione}\\[0.1cm] % Major heading such as course name
\textsc{\large M.Sc. in Computer Science and Engineering}\\[1.8cm] % Minor heading such as course title

 
%----------------------------------------------------------------------------------------

%----------------------------------------------------------------------------------------
%	TITLE SECTION
%----------------------------------------------------------------------------------------

\HRule \\[0.4cm]
{ \huge \bfseries myTaxiService} \\[0.2cm]% Title of your document
	{\Large Software Design Document}
\HRule \\[1.5cm]
 
%----------------------------------------------------------------------------------------
%	AUTHOR SECTION
%----------------------------------------------------------------------------------------


% If you don't want a supervisor, uncomment the two lines below and remove the section above
\large \emph{Authors:}\\
Angelo  \textsc{Gallarello}\\
Edoardo  \textsc{Longo}\\
Giacomo  \textsc{Locci}\\[1.5cm]

%----------------------------------------------------------------------------------------
%	DATE SECTION
%----------------------------------------------------------------------------------------

{\large \today} % Date, change the \today to a set date if you want to be precise



\vfill % Fill the rest of the page with whitespace

\end{titlepage}

\newpage

\tableofcontents

\newpage

\listoffigures

\newpage

% 0 - Abstract
\import{../main/}{abstract.tex}

% 1 - Overview
\import{../overview/}{overview.tex}
\newpage
% 2 - High level components
\import{../high-level/}{high-level.tex}
\newpage
% 3 - Component view
\import{../component-view/}{component-view.tex}
\newpage
% 4 - Deployment view
\import{../deployment-view/}{deployment-view.tex}
\newpage
% 5 - Runtime view
\import{../runtime-view/}{runtime-view.tex}
\newpage
% 6 - Component interfaces
\import{../component-interfaces/}{component-interfaces.tex}
\newpage
% 7 - Arch styles and patterns
\import{../architecture/}{architecture.tex}
\newpage
% 8 - Algorithm desing
\import{../algorithm-design/}{algorithm-design.tex}
\newpage
% 9 - User interface design
\import{../user-interface/}{user-interface.tex}
% 10 - Appendix
\import{../appendix/}{appendix.tex}




\end{document}