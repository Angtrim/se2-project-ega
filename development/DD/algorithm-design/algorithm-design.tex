%!TEX root = ../main/main.tex


\section{Algorithm design} % (fold)
\label{sec:algorithm_design}

\definecolor{keywords}{RGB}{255,0,90}
\definecolor{comments}{RGB}{0,0,113}
\definecolor{red}{RGB}{160,0,0}
\definecolor{green}{RGB}{0,150,0}

\subsection{Precedence Management Algorithm} % (fold)
\label{sub:first_algorithm}
This algorithm is created to manage the way taxi drivers are popped out from the queue.\\
The algorithm is implemented in the \textbf{QueueManager} component of the \textbf{Back-End application}.
It needs:

\begin{itemize}
	\item The list of all the available drivers, and in every driver object a reliable coordinate should be included.
	\item The possibility to interact with the Google Maps API to calculate the time to reach the position of the taxi request.
	\item A generic math library to compute the precedenceFactor.
\end{itemize}



\lstset{language=Python, 
        basicstyle=\ttfamily\small, 
        keywordstyle=\color{keywords},
        commentstyle=\color{comments},
        stringstyle=\color{red},
        showstringspaces=false,
        identifierstyle=\color{green},
        procnamekeys={def,class},
        breaklines = true}

\lstinputlisting{taxi.py}

% subsection first_algor (end)

\subsection{Shared Ride Compatibility Algorithm} % (fold)
\label{sub:second_algorithm}
This is the algorithm used to check whether two users have the possibility to share the ride.\\
The algorithm is implemented in the \textbf{RideManager} component of the \textbf{Back-End application}.
It needs:
\begin{itemize}
	\item The list of all the users that are waiting for a shared ride.
	\item The possibility to query the Google Maps API to calculate an estimated time to complete the ride.
	\item The list of all the zone objects.
\end{itemize}

Notice that if a user is still waiting in the queue, that means he/she has not found a compatible user yet, as he/she would be immediately popped out of the list in case of matching.
A timer of 10 minutes starts as the user requests the shared ride and as soon as it ends a dialog informing that a shared ride has not been found, will be shown.

\lstset{language=Python, 
        basicstyle=\ttfamily\small, 
        keywordstyle=\color{keywords},
        commentstyle=\color{comments},
        stringstyle=\color{red},
        showstringspaces=false,
        identifierstyle=\color{green},
        procnamekeys={def,class},
        breaklines = true}

\lstinputlisting{pathfinding.py}

% subsection second_algor (end)

\subsection{Zone Assignment Algorithm} % (fold)
\label{sub:third_algorithm}
This algorithm assign the taxi driver to the right zone.\\
The algorithm is implemented in the \textbf{PositionManager} component of the \textbf{Back-End application}.
It requires:
\begin{itemize}
	\item To be periodically triggered to update the membership of every taxi driver.
	\item The list of all the active taxi drivers.
	\item The list of all the zone objects.
\end{itemize}

\lstset{language=Python, 
        basicstyle=\ttfamily\small, 
        keywordstyle=\color{keywords},
        commentstyle=\color{comments},
        stringstyle=\color{red},
        showstringspaces=false,
        identifierstyle=\color{green},
        procnamekeys={def,class},
        breaklines = true}

\lstinputlisting{taxyzone.py}

% subsection third_algor (end)



% section algorithm_design (end)
