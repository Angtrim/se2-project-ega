%!TEX root = ../main/main.tex
\section{High Level Components} % (fold)
\label{sec:high_level_components}
The system could be divide in three main high level components that not necessarily correspond only to one real application:
\paragraph{Server} % (fold)
\label{comp:server}\hfill \\
The Server component is the kernel of the service we want to provide, it incorporates most of the \emph{business logic}, it stores most of the \emph{data} and it provides programmatic interfaces to the clients.
% paragraph server (end)
\paragraph{User Client} % (fold)
\label{comp:user_client}\hfill \\
The User Client components is an high level representation of the real clients available to the users of our service. It's modeled as a \emph{thin client} and it relies on the \emph{Server} to fulfill its tasks.
% paragraph user_client (end)
\paragraph{Taxi Driver Client} % (fold)
\label{par:taxi_driver_client}\hfill \\
The Taxi Driver Client component is an high level representation of the real clients available to the taxi drivers registered to the service. It's modeled as a \emph{thin client} and it relies on the \emph{Server} to fulfill its tasks.
% paragraph taxi_driver_client (end)


\subsubsection{Components Interaction} % (fold)
\label{ssub:components_interaction}
From a high level prospective the system is design following the well known \emph{client-server} paradigm.\\
The interaction between the components is handled by \nameref{comp:server} that provides a programmatic interfaces that is able to receive remote call from the clients.\\
The clients never communicates directly one with the other.
% subsubsection components_interaction (end)
% section high_level_components (end)

\section{Component View} % (fold)
\label{sec:component_view}
This section highlights the main features and roles of every component of the system.
\subsubsection{\nameref{comp:server}} % (fold)
\label{ssub:nameref_}
The \nameref{comp:server} is composed by :

\paragraph{Back-End Application} % (fold)
\label{par:back_end_application}\hfill \\
As stated in \emph{section 1.2.2} of the \emph{RASD}, the \emph{Back-End Application} is the system component that handles most of the business logic.\\
The application is written in \emph{Java EE} and to fulfill its tasks (see \emph{section 3.5.3} of the \emph{RASD}) it needs to interface with the Internet network using the \emph{HTTPS protocol} and the \emph{JAVA API for RESTful Web Service}\footnote{See \url{https://jax-rs-spec.java.net/}}, with a \emph{MySQL database} and with external Google Maps API.\\

\paragraph{MySQL Database} % (fold)
\label{par:mysql_database}\hfill \\
The MySQL database fulfill the task off storing and granting access to all the data generated and used by the service.\\
A \emph{database dump} is performed daily during the period of minor activity of the service \footnote{At first, when no activity data is available, the dump will be performed at 04:00 A.M}.\\
The connection between the \emph{Java EE} application and the databased is supported by the \emph{JDBC connector}\footnote{See \url{http://dev.mysql.com/downloads/connector/j/}}\\
% TO-DO : Connections diagram
% paragraph mysql_database (end)
% paragraph back_end_application (end)
% subsubsection nameref_ (end)
% section component_view (end)






