%!TEX root = ../RASD/main.tex

\begin{abstract}

The current document represents the \emph{Requirement Analysis and Specification Document} (RASD) of \emph{myTaxiService} system.
This document is intended to be read and used by everyone involved with the development of the software (i.e. developers, testers, project managers) and the delegating institution (i.e. the government of the large city). Therefore its purpose is to provide the intended reader with a fulfilling guideline, that can be used as the main resource for the development of the system, and with a presentation of the challenges that will have to be addressed throughout the whole process.

An incredible amount of effort has been put into every section of this document; every aspect of the \emph{system-to-be} has been thoroughly explored, meticulously analyzed and accurately studied so that every situation can be addressed in a unique way and can be interpreted without ambiguity.

In the following sections the system is broken down into its components, examining its functional and non-functional requirements. An in depth analysis of the relationship between the system and the external world is conducted throughout the document, stressing the assumptions that have been taken and highlighting the constraints that every requirement has to satisfy in order to have a functional and useful system.


The reader of this document will find UML class diagrams, sequence diagrams and use cases that provide all the needed information required to build the system.
Furthermore, every use case is justified by the goals and assumptions that lead to the need of a specific use case.

Moreover, the delegating institution will find a reason for every design choice taken and therefore is able understand the necessity of a specific feature.

This way, each and every reader will have a clear understanding of the \emph{system-to-be}, by finding a justification for every decision taken concerning any aspect that is worth of considering for a successful development of the investigated system.

\newpage

\end{abstract}

\section{Introduction}
\label{sec:introduction}

This section introduces the \emph{system-to-be}, and provides a high level description of its functioning, together with the main functionalities that the system will offer.


\subsection{Purpose}

The system is a software named \emph{myTaxiService}. It is a taxi service that will operate in a big city; the main purpose is to simplify the access of passengers to the service and to guarantee a fair management of the taxi queues.

\subsection{Scope}
\label{sub:scope}
\subsubsection{Goals} % (fold)
\label{ssub:goals}
The aim of \emph{myTaxiService} presented in the RASD is to:
\begin{enumerate} [label = \textbf{[G\arabic*]}]
\item \textbf{Intuitive service}\hfill \\
\label{goal:intuitive}
Make the process of requesting a service intuitive for the majority of potential users.
\item \textbf{Fast}\hfill \\
\label{goal:time}
Lower the time required from a user to request or book a taxi. Request time is less than 30 seconds, booking time is less than 1,5 minutes.
\item \textbf{Availability}\hfill \\
\label{goal:availability}
Enable the user to take advantage of the taxi service from a wide range of locations, namely from every location where the user has access to an Internet enabled device that can run the application.
\item \textbf{Fair queues}\hfill \\
\label{goal:queue}
Guarantee a fair management of the taxi queue by minimizing the idle time of every taxi driver using the service.
\end{enumerate}
% subsubsection goals (end)

\subsubsection{Applications} % (fold)
\label{ssub:applications}
To accommodate these goals the following pieces of software are to be developed:

%%%%%%%%%%%%%%
\paragraph{Mobile Application (\emph{\nameref{def:user}})}
\label{app:mobileuser}
  The mobile application is available for all major mobile OS (Android, iOS, Windows Mobile, Blackberry).
  This one is specifically addressed towards the \emph{\nameref{def:user}}, giving him a broad range of actions to perform.
  It gives the \emph{\nameref{def:user}} the opportunity to manage his profile, request a taxi and make or cancel a reservation. 
% subsubsection mobile_application_ (end)

\paragraph{Web Application} % (fold)
  \label{app:web}
  The web application is thought as an alternate way, for the \emph{\nameref{def:user}} to manage his profile, request a taxi and make or cancel a reservation.

\paragraph{Mobile Application (\emph{\nameref{def:taxidriver}})} % (fold)
  \label{app:mobiledriver}
  The mobile application is available for all major mobile OS (Android, iOS, Windows Mobile, Blackberry).
  This one is specifically addressed towards the \emph{\nameref{def:taxidriver}}.
  It enables the \emph{\nameref{def:taxidriver}} to accept a request issued by the system or dismiss it, signal any problem to the \emph{\nameref{def:operator}}.



\paragraph{Back-End Application} % (fold)
  \label{app:backend}
  The back-end is the application that manages the request system and queue system. This also provides a control panel used by the \emph{\nameref{def:operator}} to have an overall view of the whole system, and enables him to issue warnings to \emph{\nameref{def:user}s} or \emph{\nameref{def:taxidriver}s}, and to respond to warnings issued by the \emph{\nameref{def:taxidriver}}.
%%%%%%%%%%
% subsubsection applications (end)
%%
\subsection{Definitions}
\label{sub:def}
Here is a list of the words used in this document:

%paragraph
\paragraph{User} \hfill \\
\label{def:user} He/She is the end user of the service that, once subscribed, makes use of the taxi service as a passenger.

\paragraph{Taxi Driver} \hfill \\
\label{def:taxidriver} He/She is another kind of customer of the service. He/She is the one that, once subscribed, will be commissioned with a work.

\paragraph{Support Operator} \hfill \\
\label{def:operator} He/She is the person that is in charge of the customer service and interfaces with the customers, both the taxi drivers and the users.

\paragraph{Technician} \hfill \\
\label{def:technician} He/She is responsible for the maintenance.

\paragraph{Taxi Zone} \hfill \\
\label{def:taxi_zone} Area of the city of 2km\textsuperscript{2} .

\paragraph{Admissible Paths} \hfill \\
\label{def:admissible} Two paths are admissible if the destination point of one path is included in the other and the starting points belong to the same taxi zone.

\paragraph{Immediate Request} \hfill \\
\label{def:immediate-request} An immediate request is a request issued by the \nameref{def:user} in order to request a ride on the spot.

\paragraph{Reservation Request} \hfill \\
\label{def:reservation-request} A reservation request is a request issued by the \nameref{def:user} in order to request a reservation for a ride that will happen in the at least two hours in the future.

\paragraph{Pending Request} \hfill \\
\label{def:pending-request} A pending request is a request issued by the \nameref{def:user} that has received neither a confirmation nor a refusal.

\paragraph{Active Ride} \hfill \\
\label{def:active-ride} An active ride is a ride that had started (the taxi had departed) but not has not yet finished (the taxi has not arrived at the destination).


\subsection{References}
\begin{itemize}
\item IEEE Std 830-1998: ``IEEE Recommended Practice for Software Requirements Specifications"
\item Project description: ``Software Engineering 2 Project, AA 2015/2016"
\item Alloy Language Reference: \url{http://alloy.mit.edu/alloy/documentation/book-chapters/alloy-language-reference.pdf}
\item UML Language Reference: \url{https://www.utdallas.edu/~chung/Fujitsu/UML_2.0/Rumbaugh--UML_2.0_Reference_CD.pdf}
\end{itemize}
%%\item iOS guideline

%%\item Android guideline

\subsection{Document Overview}
This document provides a detailed description of the system and it is structured in three main sections:

\begin{description}
  \item[\nameref{sec:introduction}] \hfill \\
  This section gives a high level description of the main topics covered throughout the whole document, that is the system purpose and its scope.
  \item[\nameref{sec:overall-description}] \hfill \\
  This section offers the reader an overview over the general factors and functions that affect the product and its requirements.
  \item[\nameref{sec:specific-requirements}] \hfill \\
  This offers an insight over every functional and non functional requirement.
\end{description}