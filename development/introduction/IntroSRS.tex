\documentclass[12pt, a4paper]{article}
\begin{document}

\section{Introduction}

This is the intro.

\subsection{Purpose}

The current document represents the \emph{Requirement Analysis and Specification Document} (RASD) of \emph{myTaxiService} system.
This document is intended to be read and used by everyone involved with the development of the software (i.e. developers, testers, project managers). Therefore its purpose is to provide the intended reader with a fulfilling guideline, that can be used as the main resource for the development of the system.
An incredible amount of effort has been put into every section of this document; every aspect of the \emph{system-to-be} has been explored, analyzed and studied so that every situation can be addressed in a unique way and and can be interpreted without ambiguity.

In the following sections the system is broken down into its components, examining its functional and non-functional requirements; an in depth analysis of the relationship between the system and the external world is conducted throughout the document, stressing the assumptions that have been taken and highlighting the constraints that every requirement has to satisfy in order to have a functional and useful system.

\subsection{Scope}

The aim of the system being presented in the RASD is to:
\begin{itemize}
\item Make the process of requesting a service intuitive for any kind of user.
\item Lower the time required from a user to request or book a taxi. Request time is less than 30 seconds, booking time is less than 1,5 minutes.
\item Enable the user to take advantage of the taxi service from a wide range of locations, namely from every location where the user has access to an Internet enabled device that can run the application.
\item Guarantee a fair management of the taxi queue by minimizing the idle time of every taxi driver using the service.
\end{itemize}
To accommodate these goals the following pieces of software will have to be developed:

\begin{description}
  \item[Mobile application] \hfill \\
  The mobile application is available for all major mobile OS (Android, iOS, Windows Mobile, Blackberry).
  There are two different version of the app, one specifically addressed to the \emph{taxi user}, another one specifically addressed to the \emph{taxi driver}.

  The former gives the \emph{user} the opportunity to manage his profile, request a taxi and make or dismiss a reservation.
  The latter enables the \emph{taxi} driver to accept a request issued from the system or dismiss it, signal any problem to the operators.

  \item[Web application] \hfill \\
  The web application is thought to an alternate way for the \emph{taxi user} to manage his profile, request a taxi and make or dismiss a reservation.

  \item[Back-end] \hfill \\
  The back-end is the application that manages the request system and queue system.

  \item[Control Panel] \hfill \\
  The control panel is the application used by the \emph{operator} to have an overall view of the whole system, issue warnings to users or taxi drivers.
\end{description}

%%

\subsection{Definitions}

\subsection{References}
\begin{itemize}
\item IEEE Std 830-1998: ``IEEE Recommended Practice for Software Requirements Specifications"
\item Project description: ``Software Engineering 2 Project, AA 2015/2016"
\end{itemize}
%%\item iOS guideline

%%\item Android guideline

\subsection{Overview}
This document provides a detailed description of the system and it is structured in three main sections:

\begin{description}
  \item[Introduction] This section gives a high level description of the main topics covered throughout the whole document, that is the system purpose and its scope.
  \item[Overall description] This section offers the reader an overview over the general factors that affect the product and its requirements.
  \item[Specific requirements] This offers an insight over every functional and non functional requirement
\end{description}

\end{document}