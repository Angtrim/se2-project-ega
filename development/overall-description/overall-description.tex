%!TEX root = ../RASD/main.tex
\section{Overall Description}
\subsection{Product Perspective}
\label{sec:product_perspective}
As stated in the \emph{\nameref{sub:scope}} section the product we will release is composed by four main software applications.
\begin{itemize}
  \item A \textbf{\nameref{app:web}} addressed to the \emph{users}  to use our service. This application must interface mainly with the \textbf{\nameref{app:backend}} and with Google's Maps API.
  \item A \textbf{\nameref{app:mobileuser}} addressed to the \emph{users} to use our service and available for Android, iOS, Windows Mobile and Blackberry. This application must interface mainly with our \textbf{\nameref{app:backend}} and with Google's Maps API
  \item A \textbf{\nameref{app:mobiledriver}} addressed to the \emph{taxi drivers} to use our service and available for Android, iOS. This application must interface mainly with our \textbf{\nameref{app:backend}} and with Google's Maps API

  \item A \textbf{\nameref{app:backend}} that will handle all the business logic and that must interface mainly with Google's Maps API and with a MongoDB database .
\end{itemize}

\subsubsection{User Interfaces}
\label{subs:user_interfaces}
%to complete

\subsubsection{Hardware Interfaces}
\label{subs:hardwareinterfaces}
Both the \textbf{\nameref{app:mobileuser}} and the \textbf{\nameref{app:backend}} must interface with the GPS module and with the Network module.\\

\subsubsection{Software Interfaces}
\label{subs:softwareinterfaces}


\paragraph{\nameref{app:web}}

\begin{itemize}
	\item MyTaxyService API
	\begin{itemize}
		\item Mnemonic : Back-end
	\end{itemize}
	\item Google Maps API
	\begin{itemize}
		\item Mnemonic : Google Maps API
		\item Version Number : V3
		\item Source : \url{https://developers.google.com/maps/documentation/javascript/} 
	\end{itemize}

\end{itemize}

\paragraph{\nameref{app:mobileuser} and \nameref{app:mobiledriver}} 
\begin{itemize}
	\item MyTaxyService API
	\begin{itemize}
		\item Mnemonic : Back-end
	\end{itemize}
	\item Google Maps API
	\begin{itemize}
		\item Mnemonic : Google Maps API
		\item Version Number : V3
		\item Source : \url{https://developers.google.com/maps/} 
	\end{itemize}

	\item Android SDK
	\begin{itemize}
		\item Mnemonic : Android
		\item Version Number : 6.0
		\item Source : \url{http://developer.android.com/sdk/index.html} 
	\end{itemize}

	\item iOS SDK
	\begin{itemize}
		\item Mnemonic : iOS
		\item Version Number : 9.2
		\item Source : \url{https://developer.apple.com/ios/download/} 
	\end{itemize}

	\item Windows Mobile SDK
	\begin{itemize}
		\item Mnemonic : Windows Mobile
		\item Version Number : 6.5
		\item Source : \url{http://www.microsoft.com/en-us/download/details.aspx?id=17284} 
	\end{itemize}

	\item BlackBerry SDK
	\begin{itemize}
		\item Mnemonic : BlackBerry
		\item Version Number : 10
		\item Source : \url{https://developer.blackberry.com/} 
	\end{itemize}		
\end{itemize}

\paragraph{\nameref{app:backend}} 

\begin{itemize}
	\item Node.js API
	\begin{itemize}
		\item Mnemonic : Node.js
		\item Version : 4.2.1
		\item Source : \url{https://nodejs.org/api/}
	\end{itemize}

	\item MongoDB API
	\begin{itemize}
		\item Mnemonic : MongoDB
		\item Version : 3.0
		\item Source : \url{https://docs.mongodb.org/manual/}
	\end{itemize}

	\item Google Maps API
	\begin{itemize}
		\item Mnemonic : Google Maps API
		\item Version Number : V3
		\item Source : \url{https://developers.google.com/maps/documentation/javascript/}
	\end{itemize}
	\item Javascript API
	\begin{itemize}
		\item Mnemonic : Javascript
		\item Source : \url{https://developer.mozilla.org/en/docs/Web/API} 
	\end{itemize}
\end{itemize}

\subsubsection{Communication Interfaces} 
\label{ssub:communication_interfaces}
\paragraph{\nameref{app:web}}\mbox{} \\
Every application must interface with the Internet network. This interface is handled by the operative systems and not by The applications themselves.

\subsubsection{Memory Constrains} 

\label{ssub:memory_constrains}
\paragraph{\nameref{app:web} and \nameref{app:mobileuser} and \nameref{app:mobiledriver} }
The Mobile and Web Applications can not exceed 75MB of RAM usage.
\paragraph{\nameref{app:backend}}
The Back-End application can not exceed 15GB of total RAM usage.

\subsubsection{Site Adaptation Requirements} 
\label{ssub:site_adaptation_requirements}
\paragraph{Software Adaptation} \mbox{} \\
Every Mobile Application must be developed according to the platforms design guidelines.

\subsection{Product Functions} 
\label{sub:product_functions}
This section highlights the main product functions sorted by application.

\subsubsection{\nameref{app:web} and \nameref{app:mobileuser}}
\label{ssub:web_application_and_mobile_application_}
\begin{enumerate} [label = \textbf{[F\arabic*]}]
	\item  The app must allow the user to book a taxi on the spot.
	\item  The app must allow the user to reserve a taxi ride up to 7 days before the chosen date.
	\item  The app must allow the user to share his ride with other people whose destination is included in the ride path.
\end{enumerate}

\subsubsection{\nameref{app:mobiledriver}} 
\label{ssub:mobile_application_}
\begin{enumerate} [resume*]
	\item  The app must provide position updates to the \emph{back-end application}
	\item  The app must allow the taxi driver to accept or refuse a ride request.
	\item  The app must allow the taxi driver to report issues that could delay his arrival (traffic jam, car faults, etc ...)
	\item  The app must allow the taxi driver to overview a history of his last rides.
	\item  The app must allow the taxi driver to report a absent user.
\end{enumerate}

\subsubsection{\nameref{app:backend}} 
\label{ssub:back_end_application}
\begin{enumerate} [resume*]
	\item  The application must provide an interface for the registration process of users and taxi drivers
	\item  The application must handle the request queue using a fair policy [Rif needed]
	\item  The application must provide an accurate [rif needed] estimate of the total cost and time of a ride
	\item  The application must provide an interface to allow an operator to perform all basics CRUD operations

\end{enumerate}

\subsection{User Characteristics} 
\label{sub:user_characteristics}
This section highlights the main characteristics of the  \emph{actors} defined in section \ref{sub:def} on page \pageref{sub:def}.

\paragraph{\nameref{def:user}} The user has to be at least 14 years old and own a valid ID.\\ 
The user to be familiar with standard mobile applications interfaces.
\paragraph{\nameref{def:taxidriver}} The driver has to be a regularly licensed taxi driver.\\ The driver has to be familiar with standard mobile applications interfaces.
\paragraph{\nameref{def:technician}} 
The technician must be able to debug and identify issues in a full stack environment. The technician must be familiar with the application requirements and the overall infrastructure and design.
\paragraph{\nameref{def:operator}}  The support operator must be familiar with the overall application design and has to know how to resolve common user's issues.

\subsection{Constraints} 
\label{sub:constraints}

\subsubsection{Regulatory Policies} 
\label{ssub:regulatory_policies}
\begin{itemize}
	\item \textbf{Privacy Policy} Data should be collected and stored following the privacy policy guidelines provided by Mozzilla Foundation \url{https://developer.mozilla.org/en-US/Marketplace/Publishing/Policies_and_Guidelines/Privacy_policies}
	\item \textbf{Taxi Driver Policy} The client application for taxi drivers must be accessible only by registered and regularly licensed driver.
\end{itemize}

\subsubsection{Hardware Limitations} 
\label{ssub:hardware_limitations}
\paragraph{Mobile Application (User)} 
The Mobile Application (User) must be developed in order to be available for the 80\% of the devices currently on the market for each different platform.
\paragraph{Web Application (User)}
The Web Application must be developed in order to be available for the 100\% of devices that support the latest version of most used browser \footnote {\url{http://www.w3schools.com/browsers/browsers_stats.asp}}


\subsubsection{Interfaces to other applications} 
\label{ssub:interfaces_to_other_applications}
\paragraph{\nameref{app:mobileuser}} 
The developer of the \textbf{\nameref{app:mobileuser}} must follow the design guideline \footnote{\url{http://developer.android.com/design/index.html}} \footnote{\url{https://developer.apple.com/design/}} of the platform and has to interface with the respective Developer Console and make sure that the app will be accepted and published. \\
The application must interface with Google's Maps API \footnote{See section \ref{subs:softwareinterfaces} on page \pageref{subs:softwareinterfaces}}
\paragraph{\nameref{app:web}} 
The application must interface with Google's Maps API.
\paragraph{\nameref{app:mobiledriver}} 
The application must interface with Google's Maps API.
\paragraph{\nameref{app:backend}} 
The application must interface with Google's Maps API.

\subsubsection{Parallel Operations} 
\label{ssub:parrallel_operations}
\paragraph{\nameref{app:backend}} 
The \textbf{\nameref{app:backend}} must handle parallel request from multiple users without generating conflicts or inconsistency.\\
The system must decline a request from a user if it finds another active request from the same user.

\subsubsection{Reliability Requirements} 
\label{ssub:reliability_requirements}
\paragraph{\nameref{app:backend}} 
This requirement of reliability and availability refers to the total time the system is available for the applications of the end users.\\
The system must reach a 99.99\% uptime (corresponding to a inactivity time of 50 minutes/year).
\subsubsection{Safety and Security} 
\label{ssub:safety_and_security}
\paragraph{\nameref{app:mobiledriver}} 
The application must include an automatic night mode and gallery mode in order not to compromise the driver security.

\subsection{Assumptions And Dependencies} 
\label{sub:assumptions_and_dependencies}
\subsubsection{Domain Assumptions} 
\label{ssub:domain}
We assume that the system will be deployed in a city of about 1.5M inhabitants with about 4800 regular licensed taxi drivers.\\
We assume that at launch time the end user \textbf{\nameref{app:mobileuser}} will reach an audience of about 100k users while the \textbf{\nameref{app:web}} will reach an audience of about 10k users.\\
We assume that the city is dived in taxi zones.
\subsubsection{Taxi Driver Assumptions} 
We assume that every taxi driver will have a compatible device  \footnote{See section \ref{sub:constraints} on page \pageref{sub:constraints}} (personal or provided by the project client) with a working GPS module.
\subsubsection{User Assumptions} 
\label{ssub:user_assumptions}
We assume that every user will have a device with an enabled Internet connection.\\
We also assume that the location provided by the user (manually or automatically via GPS) is always correct.

